\documentclass{article}
\usepackage[14pt]{extsizes}
\usepackage[utf8]{inputenc}
\usepackage{setspace,amsmath}
\usepackage[left=10mm, top=10mm, right=10mm, bottom=10mm, nohead, footskip=10mm]{geometry}
\usepackage[T2A]{fontenc}
\usepackage[english,russian]{babel}
\usepackage{graphicx}
\usepackage[pdf]{graphviz}
\usepackage[table]{xcolor}
\usepackage{morewrites}
\usepackage{verbatim}
\usepackage{amsmath}
\usepackage{calc}
\usepackage{tikz}
\usepackage{pgfplots}
\usepackage{import}
\usepackage{xifthen}
\usepackage{pdfpages}
\usepackage{transparent}

\begin{document}

%ТИТУЛЬНЫЙ ЛИСТ

\begin{center}

\large{Домашнее задание}\\
\large{По предмету Теоретические Модели Вычислений}\\
\large{По теме Регулярные языки и конечные автоматы}\\

\hfill \break
\hfill \break
\hfill \break
\hfill \break
\hfill \break

\ Выполнила студентка группы А-13а-19\\
\ Филиппенко Вероника\\

\hfill \break

\end{center}

%КОНЕЦ ТИТУЛЬНОГО ЛИСТА

% ОГЛАВЛЕНИЕ

\newpage

    \tableofcontents
    
\newpage

% КОНЕЦ ОГЛАВЛЕНИЯ 
 
% ПУНКТ №1

\newpage

\section{Построение конечного автомата, распознающего заданный язык.}

\begin{enumerate}

\item {$L = \{ w \in \{a,b,c\}* $ | $ {|w|_c} = 1 \}$}\\

\ Язык состоит из всех слов, содержащих буквы $\{a, b, c\}$, причем буква $c$ содержится в единственном экземпляре. \\

\digraph[scale=1]{Z1G1}{
	rankdir=LR; 
	node [shape=none]; beg;
	node [shape=circle]; q1;
	node [shape=doublecircle]; q2;
	beg -> q1;
	q1 -> q2 [label = "c"];
	q1 -> q1 [label = "a, b"];
	q2 -> q2 [label = "a, b"];
}

\item {$L = \{ w \in \{a,b\}* $ | $ {|w|_a} \le 2, {|w|_b} \ge 2 \}$}\\ 

\ Язык состоит из всех слов, содержащих буквы $\{a, b\}$, причем букв $a$ не более двух, а букв $b$ не менее двух. \\

\digraph[scale=1]{Z1G2}{
	rankdir=LR; 
	node [shape=none]; beg;
	node [shape=circle]; q1 q2 q6 q7 q8 q9;
	node [shape=doublecircle]; q3 q4 q5;
	beg -> q1;
	q1 -> q2 [label = "b"];
	q1 -> q6 [label = "a"];
	q2 -> q3 [label = "b"];
	q2 -> q7 [label = "a"];
	q3 -> q3 [label = "b"];
	q3 -> q4 [label = "a"];
	q4 -> q4 [label = "b"];
	q4 -> q5 [label = "a"];
	q5 -> q5 [label = "b"];
	q6 -> q7 [label = "b"];
	q6 -> q8 [label = "a"];
	q7 -> q4 [label = "b"];
	q7 -> q9 [label = "a"];
	q8 -> q9 [label = "b"];
	q9 -> q5 [label = "b"];
}

\item {$L = \{ w \in \{a,b\}* $ | $ {|w|_a} \ne {|w|_b} \}$} \\

\ Язык состоит из всех слов, содержащих буквы $\{a, b\}$, причем буквы $a$ и $b$ содержатся в неравном количестве. \\

\digraph[scale=1]{Z1G3}{
	rankdir=LR; 
  node [shape=none]; beg;
  node [shape=circle]; q1;
  node [shape=doublecircle];
  beg -> q1;
  q1 -> q2 [label = "a"];
  q1 -> q3 [label = "b"];
  q2 -> q1 [label = "b"];
  q3 -> q1 [label = "a"];
}

\item {$L = \{ w \in \{a,b\}* $ | $ {ww = www} \}$} \\

\ Язык состоит из всех слов, содержащих только пустые символы. \\

\digraph[scale=1]{task14}{
	rankdir=LR; 
    node [shape=none]; beg;
	node [shape=doublecircle];
	beg -> q1;
}

\end{enumerate}

% КОНЕЦ ПУНКТА №1

% ПУНКТ №2
\section{Построение конечного автомата, используя прямое произведение.}

\begin{enumerate}

\item {$L_1 = \{ w \in \{a,b\} $ | $ {|w|_a} \ge 2 \wedge {|w|_b} \ge 2 \}$}\\

\ Язык состоит из всех слов, содержащих буквы $\{a, b\}$, причем обеих букв более двух. 

\ Разбиваем на два автомата: \\

\ Автомат №1: \\

\digraph[scale=1]{Z2G1P1}{
	rankdir=LR; 
	node [shape=none]; beg;
	node [shape=doublecircle]; q3;
	node [shape=circle];
	beg -> q1;
	q1 -> q1 [label = "b"];
	q1 -> q2 [label = "a"];
	q2 -> q2 [label = "b"];
	q2 -> q3 [label = "a"];
	q3 -> q3 [label = "a,b"];
}

$ \Sigma_{1} = \{ a, b \}$ \\
$ Q_{1} = \{ q1, q2, q3 \} $ \\
$ s_{1} = \{ q1 \} $ \\
$ T_{1} = \{ q3 \}$ \\

\ Автомат №2: \\

\digraph[scale=1]{Z2G1P2}{
	rankdir=LR; 
	node [shape=none]; beg;
	node [shape=doublecircle]; p3;
	node [shape=circle];
	beg -> p1;
	p1 -> p1 [label = "a"];
	p1 -> p2 [label = "b"];
	p2 -> p2 [label = "a"];
	p2 -> p3 [label = "b"];
	p3 -> p3 [label = "a,b"];
}

$ \Sigma_{2} = \{ a, b \}$ \\
$ Q_{2} = \{ p1, p2, p3 \} $ \\
$ s_{2} = \{ p1 \} $ \\
$ T_{2} = \{ p3 \}$ \\

\ Итог: \\

$ \Sigma = \Sigma_{1} \cup \Sigma_{2} = \{ a, b \}$ \\
$ Q = Q_{1} \times Q_{1} = \{ <q1,p1>, <q1,p2>, <q1,p3>, <q2,p1>, <q2,p2>, <q2,p3>, <q3,p1>, <q3,p2>, <q3,p3> \}$ \\
$ s = <s_{1}, s_{2}> = \{ <q1,p1> \}$ \\
$ T = T_{1} \times T_{2} = \{ <q3,p3> \}$ \\

\\

\ Таблица состояний: \\

\begin{table}[!htbp]

\rowcolors{1}{white}

\begin{tabular}{|l|l|l|}

\hline
  $-$     &   $a$    &   $b$   \\ \hline
<q1,p1>   & <q2,p1>  & <q1,p2>  \\ \hline
<q1,p2>   & <q2,p2>  & <q1,p3>  \\ \hline
<q1,p3>   & <q2,p3>  & <q1,p3>  \\ \hline
<q2,p1>   & <q3,p1>  & <q2,p2>  \\ \hline
<q2,p2>   & <q3,p2>  & <q2,p3>  \\ \hline
<q2,p3>   & <q3,p3>  & <q2,p3>  \\ \hline
<q3,p1>   & <q3,p1>  & <q3,p2>  \\ \hline
<q3,p2>   & <q3,p2>  & <q3,p3>  \\ \hline
<q3,p3>   & <q3,p3>  & <q3,p3>  \\ \hline

\end{tabular}

\end{table}

\ Результирующий автомат: \\

\digraph[scale=0.9]{Z2G1P0}{
	rankdir=LR; 
	node [shape=none]; beg;
	node [shape=doublecircle]; q3p3;
	node [shape=circle];
	beg -> q1p1;
	q1p1 -> q1p2 [label = "b"];
	q1p1 -> q2p1 [label = "a"];
	q1p2 -> q1p3 [label = "b"];
	q1p2 -> q2p2 [label = "a"];
	q1p3 -> q1p3 [label = "b"];
	q1p3 -> q2p3 [label = "a"];
	q2p1 -> q2p2 [label = "b"];
	q2p1 -> q3p1 [label = "a"];
	q2p2 -> q2p3 [label = "b"];
	q2p2 -> q3p2 [label = "a"];
	q2p3 -> q2p3 [label = "b"];
	q2p3 -> q3p3 [label = "a"];
	q3p1 -> q3p2 [label = "b"];
	q3p1 -> q3p1 [label = "a"];
	q3p2 -> q3p3 [label = "b"];
	q3p2 -> q3p2 [label = "a"];
	q3p3 -> q3p3 [label = "b"];
}

\\

\item {$L_2 = \{ w \in \{a,b\}* $ | $ {|w|} \ge 3 \wedge {|w|} $ is odd $ \}$}\\

\ Разбиваем на два автомата: \\

\ Автомат №1: \\

\digraph[scale=1]{Z2G2P1}{
	rankdir=LR; 
	node [shape=none]; beg;
	node [shape=doublecircle]; q4;
	node [shape=circle];
	beg -> q1;
  q1 -> q1 [label = "b"];
	q1 -> q2 [label = "a"];
  q2 -> q2 [label = "b"];
	q2 -> q3 [label = "a"];
  q3 -> q3 [label = "b"];
	q3 -> q4 [label = "a"];
	q4 -> q4 [label = "a"];
}

$ \Sigma_{1} = \{ a, b \}$ \\
$ Q_{1} = \{ q1, q2, q3, q4 \} $ \\
$ s_{1} = \{ q1 \} $ \\
$ T_{1} = \{ q4 \}$ \\

\ Автомат №2: \\

\digraph[scale=1]{Z2G2P2}{
	rankdir=LR; 
	node [shape=none]; beg;
	node [shape=circle]; p1;
	node [shape=doublecircle];
	beg -> p1;
  p1 -> p1 [label = "a"];
	p1 -> p2 [label = "b"];
	p2 -> p1 [label = "b"];
  p2 -> p2 [label = "a"];
}

$ \Sigma_{2} = \{ a, b \}$ \\
$ Q_{2} = \{ p1, p2 \} $ \\
$ s_{2} = \{ p1 \} $ \\
$ T_{2} = \{ p2 \}$ \\

\ Итог: \\

$ \Sigma = \Sigma_{1} \cup \Sigma_{2} = \{ a, b \}$ \\
$ Q = Q_{1} \times Q_{1} = \{ <q1,p1>, <q1,p2>, <q2,p1>, <q2,p2>, <q3,p1>, <q3,p2>, <q4,p1>, <q4,p2> \}$ \\
$ s = <s_{1}, s_{2}> = \{ <q1,p1> \}$ \\
$ T = T_{1} \times T_{2} = \{ <q4,p2> \}$ \\

\ Таблица состояний: \\

\begin{table}[!htbp]

\rowcolors{1}{white}

\begin{tabular}{|l|l|l|}

\hline

  $-$     &   $a$    &   $b$   \\ \hline
<q1,p1>   & <q2,p1>  & <q1,p2>  \\ \hline
<q1,p2>   & <q2,p2>  & <q1,p1>  \\ \hline
<q2,p1>   & <q3,p1>  & <q2,p2>  \\ \hline
<q2,p2>   & <q3,p2>  & <q2,p1>  \\ \hline
<q3,p1>   & <q4,p1>  & <q3,p2>  \\ \hline
<q3,p2>   & <q4,p2>  & <q3,p1>  \\ \hline
<q4,p1>   & <q4,p1>  & <q4,p2>  \\ \hline
<q4,p2>   & <q4,p2>  & <q4,p1>  \\ \hline

\end{tabular}

\end{table}

\ Результирующий автомат: \\

\digraph[scale=0.6]{Z2G2P0}{
	rankdir=LR; 
	node [shape=none]; beg;
	node [shape=doublecircle]; q4p2;
	node [shape=circle];
	beg -> q1p1;
	q1p1 -> q1p2 [label = "b"];
	q1p1 -> q2p1 [label = "a"];
	q1p2 -> q1p1 [label = "b"];
	q1p2 -> q2p2 [label = "a"];
	q2p1 -> q2p2 [label = "b"];
	q2p1 -> q3p1 [label = "a"];
	q2p2 -> q2p1 [label = "b"];
	q2p2 -> q3p2 [label = "a"];
	q3p1 -> q4p1 [label = "a"];
	q3p1 -> q3p2 [label = "b"];
	q3p2 -> q3p1 [label = "b"];
	q3p2 -> q4p2 [label = "a"];
	q4p1 -> q4p2 [label = "b"];
	q4p1 -> q4p1 [label = "a"];
	q4p2 -> q4p1 [label = "b"];
	q4p2 -> q4p2 [label = "a"];
}

\item {$L_3 = \{ w \in \{a,b\}* $ | $ {|w|_a} $ is even $ \wedge {|w|_b} $ multiple of 3 $ \}$}\\

\ Разбиваем на два автомата: \\

\ Автомат №1: \\

\digraph[scale=1]{Z2G3P1}{
	rankdir=LR; 
	node [shape=none]; beg;
	node [shape=circle];q2
	node [shape=doublecircle];
	beg -> q1;
	q1 -> q1 [label = "b"];
	q1 -> q2 [label = "a"];
	q2 -> q1 [label = "a"];
	q2 -> q2 [label = "b"];
}

$ \Sigma_{1} = \left\{ a, b \right\}$ \\
$ Q_{1} = \left\{ q1, q2 \right\} $ \\
$ s_{1} = \left\{ q1 \right\} $ \\
$ T_{1} = \left\{ q1 \right\}$ \\

\ Автомат №2: \\

\digraph[scale=1]{Z2G2P2}{
	rankdir=LR; 
	node [shape=none]; beg;
	node [shape=doublecircle]; p1;
	node [shape=circle];
	beg -> p1;
	p1 -> p1 [label = "a"];
	p1 -> p2 [label = "b"];
	2p -> p2 [label = "a"];
	2p -> p3 [label = "b"];
	p3 -> p3 [label = "a"];
	p3 -> p1 [label = "b"];
}

$ \Sigma_{2} = \left\{ a, b \right\}$ \\
$ Q_{2} = \left\{ p1, p2, p3 \right\} $ \\
$ s_{2} = \left\{ p1 \right\} $ \\
$ T_{2} = \left\{ p1 \right\}$ \\

\ Итог: \\

$ \Sigma = \Sigma_{1} \cup \Sigma_{2} = \{ a, b \}$ \\
$ Q = Q_{1} \times Q_{1} = \{ <q1,p1>, <q1,p2>, <q1,p3>, <q2,p1>, <q2,p2>, <q2,p3> \}$ \\
$ s = <s_{1}, s_{2}> = \{ <q1,p1> \}$ \\
$ T = T_{1} \times T_{2} = \{ <q1,p1> \}$ \\

\ Таблица состояний: \\

\begin{table}[!htbp]

\rowcolors{1}{white}

\begin{tabular}{|l|l|l|}

\hline

  $-$     &   $a$    &   $b$   \\ \hline
<q1,p1>   & <q2,p1>  & <q1,p2>  \\ \hline
<q1,p2>   & <q2,p2>  & <q1,p3>  \\ \hline
<q1,p3>   & <q2,p3>  & <q1,p1>  \\ \hline
<q2,p1>   & <q1,p1>  & <q2,p2>  \\ \hline
<q2,p2>   & <q1,p2>  & <q2,p3>  \\ \hline
<q2,p3>   & <q1,p3>  & <q2,p1>  \\ \hline

\end{tabular}

\end{table}

\ Результирующий автомат: \\

\digraph[scale=0.7]{Z2G3P0}{
	rankdir=LR; 
	node [shape=none]; beg;
	node [shape=doublecircle]; q1p1;
	node [shape=circle];
	beg -> q1p1;
	q1p1 -> q1p2 [label = "b"];
	q1p1 -> q2p1 [label = "a"];
	q1p2 -> q1p3 [label = "b"];
	q1p2 -> q2p2 [label = "a"];
	q1p3 -> q1p1 [label = "b"];
	q1p3 -> q2p3 [label = "a"];
	q2p1 -> q2p2 [label = "b"];
	q2p1 -> q1p1 [label = "a"];
	q2p2 -> q2p3 [label = "b"];
	q2p2 -> q1p2 [label = "a"];
	q2p3 -> q2p1 [label = "b"];
	q2p3 -> q1p3 [label = "a"];
}

\item {$L_4$ = $\overline{L_3}$}\\

\digraph[scale=0.7]{Z2G4}{
	rankdir=LR; 
	node [shape=none]; beg;
	node [shape=circle]; q1p1;
	node [shape=doublecircle];
	beg -> q1p1;
	q1p1 -> q1p2 [label = "b"];
	q1p1 -> q2p1 [label = "a"];
	q1p2 -> q1p3 [label = "b"];
	q1p2 -> q2p2 [label = "a"];
	q1p3 -> q1p1 [label = "b"];
	q1p3 -> q2p3 [label = "a"];
	q2p1 -> q2p2 [label = "b"];
	q2p1 -> q1p1 [label = "a"];
	q2p2 -> q2p3 [label = "b"];
	q2p2 -> q1p2 [label = "a"];
	q2p3 -> q2p1 [label = "b"];
	q2p3 -> q1p3 [label = "a"];
}

\item {$L_5 = L_2 \setminus L_3 $}\\

$L_5 = L_2 \cap \overline{L_3} $

\ Автомат $ L_{2} $ : \\

\digraph[scale=1]{Z2G5P1}{
	rankdir=LR; 
	node [shape=none]; beg;
	node [shape=doublecircle]; q4;
	node [shape=circle];
	beg -> q1;
	q1 -> q2 [label = "a,b"];
	q2 -> q3 [label = "a,b"];
	q3 -> q4 [label = "a,b"];
	q4 -> q3 [label = "a,b"];
}

\ Автомат $ \overline{L_{3}} $ :\\

\digraph[scale=0.8]{Z2G5P2}{
	rankdir=LR; 
	node [shape=none]; beg;
	node [shape=circle]; p1;
	node [shape=doublecircle];
	beg -> p1;
	p1 -> p2 [label = "b"];
	p1 -> p4 [label = "a"];
	p2 -> p3 [label = "b"];
	p2 -> p5 [label = "a"];
	p3 -> p1 [label = "b"];
	p3 -> p6 [label = "a"];
	p4 -> p5 [label = "b"];
	p4 -> p1 [label = "a"];
	p5 -> p6 [label = "b"];
	p5 -> p2 [label = "a"];
	p6 -> p4 [label = "b"];
	p6 -> p3 [label = "a"];
}

\ Результирующий автомат: \\

\digraph[scale=0.4]{Z2G5P0}{
	rankdir=LR; 
	node [shape=none]; beg;
	node [shape=doublecircle]; q4p2 q4p3 q4p5 q4p6;
	node [shape=circle];
	beg -> q1p1;
	q1p1 -> q2p2 [label="b"]
	q1p1 -> q2p4 [label="a"]
	q2p1 -> q3p2 [label="b"]
	q2p1 -> q3p4 [label="a"]
	q2p2 -> q3p3 [label="b"]
	q2p2 -> q3p5 [label="a"]
	q2p4 -> q3p1 [label="a"]
	q2p4 -> q3p5 [label="b"]
	q3p1 -> q4p2 [label="b"]
	q3p1 -> q4p4 [label="a"]
	q3p2 -> q4p3 [label="b"]
	q3p2 -> q4p5 [label="a"]
	q3p3 -> q4p1 [label="b"]
	q3p3 -> q4p6 [label="a"]
	q3p4 -> q4p1 [label="a"]
	q3p4 -> q4p5 [label="b"]
	q3p5 -> q4p2 [label="a"]
	q3p5 -> q4p6 [label="b"]
	q3p6 -> q4p3 [label="a"]
	q3p6 -> q4p4 [label="b"]
	q4p1 -> q3p2 [label="b"]
	q4p1 -> q3p4 [label="a"]
	q4p2 -> q3p3 [label="b"]
	q4p2 -> q3p5 [label="a"]
	q4p3 -> q3p1 [label="b"]
	q4p3 -> q3p6 [label="a"]
	q4p4 -> q3p1 [label="a"]
	q4p4 -> q3p5 [label="b"]
	q4p5 -> q3p2 [label="a"]
	q4p5 -> q3p6 [label="b"]
	q4p6 -> q3p3 [label="a"]
	q4p6 -> q3p4 [label="b"]
}

\ Таблица состояний: \\

\begin{table}[!htbp]

\rowcolors{1}{white}

\begin{tabular}{|l|l|l|}

\hline

  $-$     &   $a$    &   $b$   \\ \hline
<q1,p1>   & <q2,p4>  & <q2,p2>  \\ \hline
<q1,p2>   & <q2,p5>  & <q2,p3>  \\ \hline
<q1,p3>   & <q2,p6>  & <q2,p1>  \\ \hline
<q1,p4>   & <q2,p1>  & <q2,p5>  \\ \hline
<q1,p5>   & <q2,p2>  & <q2,p6>  \\ \hline
<q1,p6>   & <q2,p3>  & <q2,p4>  \\ \hline
<q2,p1>   & <q3,p4>  & <q3,p2>  \\ \hline
<q2,p2>   & <q3,p5>  & <q3,p3>  \\ \hline
<q2,p3>   & <q3,p6>  & <q3,p1>  \\ \hline
<q2,p4>   & <q3,p1>  & <q3,p5>  \\ \hline
<q2,p5>   & <q3,p2>  & <q3,p6>  \\ \hline
<q2,p6>   & <q3,p3>  & <q3,p4>  \\ \hline
<q3,p1>   & <q4,p4>  & <q4,p2>  \\ \hline
<q3,p2>   & <q4,p5>  & <q4,p3>  \\ \hline
<q3,p3>   & <q4,p6>  & <q4,p1>  \\ \hline
<q3,p4>   & <q4,p1>  & <q4,p5>  \\ \hline
<q3,p5>   & <q4,p2>  & <q4,p6>  \\ \hline
<q3,p6>   & <q4,p3>  & <q4,p4>  \\ \hline
<q4,p1>   & <q3,p4>  & <q3,p2>  \\ \hline
<q4,p2>   & <q3,p5>  & <q3,p3>  \\ \hline
<q4,p3>   & <q3,p6>  & <q3,p1>  \\ \hline
<q4,p4>   & <q3,p1>  & <q3,p5>  \\ \hline
<q4,p5>   & <q3,p2>  & <q3,p6>  \\ \hline
<q4,p6>   & <q3,p3>  & <q3,p4>  \\ \hline

\end{tabular}

\end{table}


\end{enumerate}

% КОНЕЦ ПУНКТА №2

% ПУНКТ №3

\section{Построение минимального ДКА по регулярному выражению.}

\begin{enumerate}

\item \((ab + aba)^{*}a\) \\

\ Недетерминированный автомат: \\

\digraph[scale=1]{Z3G1Px}{
    rankdir=LR; 
    node [shape = none]; beg;
    node [shape = doublecircle]; q2;
    node [shape = circle];
    beg -> q1;
    q1 -> q2 [label = "a"];
    q1 -> q3 [label = "a"];
    q1 -> q4 [label = "a"];
    q3 -> q1 [label = "b"];
    q4 -> q5 [label = "b"];
    q5 -> q1 [label = "a"];
}

\ Таблица преобразования НКА в ДКА: \\

\begin{table}[!htbp]

\rowcolors{1}{white}

\begin{tabular}{|l|l|l|}

\hline

     $-$    &     $a$     &  $b$  \\ \hline
     q1     &  q2,q3,q4   &   -   \\ \hline
  q2,q3,q4  &      -      & q1,q5 \\ \hline
   q1,q5    & q1,q2,q3,q4 &   -   \\ \hline
q1,q2,q3,q4 &  q2,q3,q4   & q1,q5 \\ \hline

\end{tabular}

\end{table}

\ Детерминированный автомат: \\

\digraph[scale=1]{Z3G1}{
    rankdir=LR; 
    node [shape = none]; beg;
    node [shape = circle]; q1 q1q5;
    node [shape = doublecircle];
    beg -> q1;
    q1 -> q2q3q4 [label = "a"];
    q2q3q4 -> q1q5 [label = "b"];
    q1q5 -> q1q2q3q4 [label = "a"];
    q1q2q3q4 -> q2q3q4 [label = "b"];
    q1q2q3q4 -> q1q5 [label = "b"];
} 

\item \(a(a(ab)^{*}b)^{*}(ab)^{*}\)\\

\ Недетерминированный автомат: \\

\digraph[scale=1]{Z3G2Px}{
    rankdir=LR;
    node [shape = none]; beg;
    node [shape = doublecircle]; q2 q5;
    node [shape = circle];
    beg -> q1;
    q1 -> q2 [label = "a"];
    q2 -> q3 [label = "a"];
    q2 -> q5 [label = "a"];
    q3 -> q4 [label = "a"];
    q4 -> q3 [label = "b"];
    q3 -> q2 [label = "b"];
    q5 -> q2 [label = "b"];
}  

\ Таблица преобразования НКА в ДКА: \\

\begin{table}[!htbp]

\rowcolors{1}{white}

\begin{tabular}{|l|l|l|}

\hline

 $-$  &  $a$  & $b$ \\ \hline
 q1   &   q2  & -   \\ \hline
q1,q2 & q3,q5 & -   \\ \hline
q3,q5 &   q4  & q2  \\ \hline
 q4   &   -   & q3  \\ \hline
 q3   &   q4  & q2  \\ \hline

\end{tabular}

\end{table}

\ Детерминированный автомат: \\

\digraph[scale=1]{Z3G2P1}{
    rankdir=LR;
    node [shape = none]; beg;
    node [shape = doublecircle]; q2;
    node [shape = circle];
    beg -> q1;
    q1 -> q2 [label = "a"];
    q2 -> q3q5 [label = "a"];
    q3q5 -> q4 [label = "a"];
    q3q5 -> q2 [label = "b"];
    q4 -> q3 [label = "b"];
    q3 -> q4 [label = "a"];
    q3 -> q2 [label = "b"];
}  

\ Минимизированный детерминированный автомат: \\

\digraph[scale=1]{Z3G2P2}{
    rankdir=LR;
    node [shape = none]; beg;
    node [shape = doublecircle]; q2;
    node [shape = circle];
    beg -> q1;
    q1 -> q2 [label = "a"];
    q2 -> q3q3q5 [label = "a"];
    q3q3q5 -> q4 [label = "a"];
    q3q3q5 -> q2 [label = "b"];
    q4 -> q3q3q5 [label = "b"];
} 

\item \((a + (a + b)(a + b)b)^*\)\\

\ Недетерминированный автомат: \\

\digraph[scale=1]{Z3G3Px}{
    rankdir=LR; 
    node [shape = none]; beg;
    node [shape = doublecircle]; q1;
    node [shape = circle];
    beg -> q1;
    q1 -> q1 [label = "a"];
    q1 -> q2 [label = "a, b"];
    q2 -> q3 [label = "a, b"];
    q3 -> q1 [label = "b"];
}  

\ Таблица преобразования НКА в ДКА: \\

\begin{table}[!htbp]

\rowcolors{1}{white}

\begin{tabular}{|l|l|l|}

\hline

   $-$   &   $a$    &   $b$   \\ \hline
   q1    &   q1,q2  &    q2    \\ \hline
  q1,q2  & q1,q2,q3 &   q2,q3  \\ \hline
   q2    &    q3    &    q3    \\ \hline
q1,q2,q3 & q1,q2,q3 & q1,q2,q3 \\ \hline
  q2,q3  &    q3    &   q1,q2  \\ \hline
   q3    &     -    &    q1    \\ \hline
  q1,q3  &   q1,q2  &   q1,q2  \\ \hline

\end{tabular}

\end{table}

\ Детерминированный автомат: \\

\digraph[scale=1]{Z3G3}{
    rankdir=LR;
    node [shape = none]; beg;
    node [shape = circle]; q2 q3 q2q3;
    node [shape = doublecircle];
    beg -> q1;
    q1 -> q1q2 [label = "a"];
    q1 -> q2 [label = "b"];
    q1q2 -> q1q2q3 [label = "a"];
    q1q2 -> q2q3 [label = "b"];
    q2 -> q3 [label = "a, b"];
    q1q2q3 -> q1q2q3 [label = "a, b"];
    q2q3 -> q3 [label = "a"];
    q2q3 -> q1q3 [label = "b"];
    q3 -> q1 [label = "b"];
    q1q3 -> q1q2 [label = "a, b"];
}  

\item \((b+c)((ab)^*c + (ba)^*)^*\)\\

\ Детерминированный автомат: \\

\digraph[scale=0.7]{Z3G4}{
    rankdir=LR;
    node [shape = none]; beg;
    node [shape = doublecircle]; q2 q5 q7;
    node [shape = circle];
    beg -> q1;
    q1 -> q2 [label = "b, c"];
    q2 -> q3 [label = "a"];
    q2 -> q5 [label = "c"];
    q2 -> q6 [label = "b"];
    q3 -> q4 [label = "b"];
    q4 -> q3 [label = "a"];
    q4 -> q5 [label = "c"];
    q5 -> q3 [label = "a"];
    q5 -> q5 [label = "c"];
    q5 -> q6 [label = "b"];
    q6 -> q7 [label = "a"];
    q7 -> q3 [label = "a"];
    q7 -> q5 [label = "c"];
    q7 -> q6 [label = "b"];
}

\end {enumerate}

% КОНЕЦ ПУНКТА №3

% ПУНКТ №4

\section{Определение, является ли язык регулярным или нет.}

\begin{enumerate}

\item \(L=\{(aab)^{n}b(aba)^{m} : n \geqslant 0, m \geqslant 0\}\)\\

\ Т.к. по этому языку можно составить ДКА, он является регулярным: \\

\digraph[scale=0.6]{Z4G1}{
    rankdir=LR;
    node [shape = none]; beg;
    node [shape = doublecircle]; q5 q8;
    node [shape = circle];
    beg -> q1;
    q1 -> q2 [label = "a"];
    q1 -> q4 [label = "lambda"];
    q2 -> q3 [label = "a"];
    q3 -> q4 [label = "b"];
    q4 -> q2 [label = "a"];
    q4 -> q5 [label = "b"];
    q5 -> q6 [label = "a"]; 
    q5 -> q8 [label = "lambda"]; 
    q6 -> q7 [label = "b"];
    q7 -> q8 [label = "a"];
    q8 -> q5 [label = "lambda"];
}  

\item \(L = \{uaav | u \in \{a, b\}^*, v \in \{a, b\}^*, |u|_b \geqslant |v|_a\}\)\\

\ Применяется лемма о разрастании. Фиксируется \(\forall n \in \mathbb{N} \), затем, разбирается слово \(\omega = b^{n}aaa^{n}, \; |\omega| = 2n + 2 \geq n\). После - все разбиения данного слова \(\omega = xyz\) такие, что \(|y| \neq 0, \; |xy| \leq n\): \\

$x = b^{k}, \; y = b^{l}, \; z = b^{n - k - l}aaa^{n}$ , где $ \(1 \leq k + l \leq n \; \wedge \; l > 0\) $

\ Других допустимых разбиений - нет. \\
\ Для любых подобных разбиений \(xy^{0}z \notin L\). => лемма не выполняется => \(L\) не является регулярным языком. \\

\item \(L = \{a^mw : w \in \{a, b\}^{*}, \; 1 \geqslant |w|_b \geqslant m\}\)\\

\ Применяется лемма о разрастании. Фиксируется \(\forall n \in \mathbb{N} \), затем разбирается слово \(\omega = a^nb^n, \; |\omega| = 2n \geqslant n\). После - все разбиения данного слова \(\omega = xyz\) такие, что \(|y| \neq 0, \; |xy| \leq n\): \\

$x = a^{l}, \; y = a^{m}, \; z = a^{n-l-m}b^{n} $ , где $ \(l + k \leqslant n \; \wedge \; m \ne 0\) $

\ Других допустимых разбиений - нет.\\ 

\ Выполняется накачка: \\

$xy^{i}z = a^{l}(a^{m})^{i}a^{n-l-m}b^{n} = a^{n-mi}b^{n} \notin L, \; i \geqslant 0 \in \mathbb{N} $

\ Лемма не выполняется => \(L\) не является регулярным языком. \\

\item \(L = \{a^{k}b^{m}a^{n} : k = n \vee m > 0\}\)\\

\ Применяется лемма о разрастании. Фиксируется \(\forall n \in \mathbb{N} \), затем, разбирается слово \(\omega = a^nba^n, \; |\omega| = 2n + 1 \geqslant n\). После - все разбиения данного слова \(\omega = xyz\) такие, что \(|y| \neq 0, \; |xy| \leq n\): \\

$x = a^{k}, \; y = a^{m}, \; z = a^{n-k-m}ba^{n} $, где $ \(k + m \leqslant n \; \wedge \; m \ne 0\) $

\ Других допустимых разбиений - нет. \\ 

\ Выполняется накачка: \\ 

$xy^{i}z = a^{k}(a^{m})^{i}a^{n-k-m}ba^{n} = a^{n+m(i-1)}ba^{n} \notin L, \; i \geqslant 2 \in \mathbb{N} $

\ Противоречие => лемма не выполняется => \(L\) не является регулярным языком. \\

\item \(L = \{ucv : u \in \{a, b\}^*, \; v \in \{a, b\}^*, u \ne v^R \}\)\\

\  Применяется лемма о разрастании. Фиксируется \(\forall n \in \mathbb{N} \), затем, разбирается слово \(\omega = (ab)^nc(ab)^n = \alpha_1\alpha_2...\alpha_{4n+1}, \; |\omega| = 4n + 1 \geqslant n\). Следом, разбираются все разбиения слова \(\omega = xyz\) , с условием \(|y| \neq 0, \; |xy| \leq n\): \\

$x = \alpha_1\alpha_2...\alpha_k, \; y = \alpha_{k+1}...\alpha_{k+m}, \; z = \alpha_{k+m+1}...\alpha_{4n+1}c(ab)^n $, где $ \(k + m \leqslant n \; \wedge \; m \ne 0\) $

\ Других допустимых разбиений - нет. \\ 

\ Выполняется накачка: \\

$xy^{i}z = (\alpha_1\alpha_2...\alpha_k)(\alpha_{k+1}...\alpha_{k+m})^i(\alpha_{k+m+1}...\alpha_{4n+1}c(ab)^n)$

\ При \(i = 2\) : \\

$xy^{2}z = (\alpha_1\alpha_2...\alpha_k)(\alpha_{k+1}...\alpha_{k+m})^2(\alpha_{k+m+1}...\alpha_{4n+1}c(ab)^n) \notin L$

\ Лемма не выполняется => \(L\) не является регулярным языком. \\

\end {enumerate}

\end{document}